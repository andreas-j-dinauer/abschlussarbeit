\section{Anforderungserhebung}
Zunächst müssen die Anforderungen an das System erhoben und kategorisiert werden. Dabei wird zwischen funktionalen und nicht-funktionalen Anforderungen unterschieden. Alle in diesem Kapitel erhobenen Anforderungen sind so gestaltet, dass die Applikation im beliebigem Kontext  und dynamisch eingesetzt werden kann. Den Anforderungen liegt kein spezifischer Anwendungsfall zugrunde.\\
Es gibt vier verschiedene Ansätze zur Erfassung für Anforderungen. Die folgenden Anforderungen wurde größtenteils auf Basis eigener Überlegungen und ersten Code-Ausführungen durch den Ansatz "Kreativität" erhoben. Für eine Beobachtung oder das Untersuchen von Artifakten sind keine Ressourcen wie z.B. Dokumente oder bereits bestehende Anforderungen verfügbar.[978-3-658-37194-4, S.60]. \\
Jeder Stichpunkt beschreibt eine Anforderung an das System, einer Komponente des Systems oder einen Ablauf im System und wird nach einem bestimmten Schema formuliert. Einer Anforderung kann eine bestimmte Verbindlichkeit zugeordnet werden, welche durch die Formulierung der Anforderung deutlich wird. Muss eine Anforderung umgesetzt werden, so hat diese die höchste Priorität und ist für die grundlegende Fertigstellung der Applikation unerlässlich. Soll eine Anforderung umgesetzt werden, so schafft diese einen deutlichen Mehrwert und verbessert die Applikation signifikant. Kann eine Anforderung umgesetzt werden, so handelt es sich dabei um eine optionale Erweiterung, welche nur umgesetzt wird, wenn die notwendigen Resourcen verfügbar sind.
\subsection{Funktionale Anforderungen}
Funktionale Anforderungen beschreiben die fachliche Spezifikation eines Systems, sowie alle Schnittstellen inklusive der Eingangs- sowie Ausgabeparameter des Systems. Diese Anforderungen werden in der Testphase der Entwicklung in Testfälle überführt. Können die Tests erfolgreich ausgeführt werden, so erfüllt die Applikation die zuvor festgelegten Anforderungen.\\
\begin{itemize}
  \item Das System soll genau eine Jira-Cloud Instanz als Datenquelle verwenden. 
  \item Die Interaktion mit der Jira-Cloud API in Java soll mittels der breits von Atlassian bereitgestellten Dependencies umgesetzt werden. Es soll Basic-Authorization verwendet werden. Als Java Build-Tool soll Maven verwendet werden.
  \item Es soll möglich sein, diese Projekte genau zu konfigurieren. Dabei soll es auch möglich sein, für jedes Projekt die zu extrahierenden Vorgangstypen festzulegen. Der Speicherort zu dieser Konfigurationsdatei muss durch Umgebungsvariablen dynamisch konfigurierbar sein. Folgende JSON-Datei gibt unserer Applikation die Anweisung, den Issuetype Task und Bug des Projektes mit dem Schlüssel KAW und den Issuetype Story und Task des Projektes mit dem Schlüssel MUN zu extrahieren und in das Zielsystem zu laden:
    \begin{lstlisting}
    {
        "KAW": ["Task", "Bug"],
        "MUN": ["Story", "Task"]
    }
    \end{lstlisting}
  \item Das System soll dynamisch an durch das Konfigurieren der Datenbankverbindung mittels Umgebungsvariablen an die Zieldatenbank angebunden werden. Dabei soll der Name und das Passwort des Benutzers, sowie der Host, der Port und das Schema der Datenbank konfiguriert werden können.
  \item Die Applikation muss den ETL-Prozess beim Start der Anwendung durchführen und nach Beendigung des ETL-Prozess mit einem Fehlercode terminieren.
  \item Das System soll nicht eigenständig die zeitliche Ausführung terminieren sondern von seiner Umgebung zur Ausführung gebracht werden. Beispielsweise kann die Terminierung in einem Kubernetes Cluster stattfinden. Ein Vorteil wäre, dass in diesem Fall mehrere Knoten zur Verfügung stehen, welche eine höhere Ausfallsicherheit garantieren. Des weiteren haben sich externe Scheduling-Methoden im Gegensatz zur selbständigen Implementierung als sehr zuverlässig erwiesen. \\
  \item Das System soll alle neuen oder geänderten Objekte periodisch exportieren. Ein Exportvorgang soll einmal pro Tag stattfinden. Ein Exportvorgang soll immer zur gleichen Tageszeit und außerhalb der Zeit geschehen, in welcher mit dem Zielsystem gearbeitet wird.
  \item Das System soll alle relevanten Objekte exportieren, die ein Jira-Projekt umfassend beschreiben. Diese sind z.B. Vorgänge, Vorgangstypen, Projekte und Kommentare.
  \item Das System muss alle Daten auf die Zieldatenbank transformieren. Alle für die Analyse nicht relevanten Metadaten aus dem Jira-System sollen aussortiert werden. Ausschließlich wichtige Felder für die folgende Analysen sollen in ein angemessenes Format transormiert werden.
  \item Die Extraktion eines jeden Objektes soll dokumentiert werden. Hierzu werden alle Vorgänge in einer Datenbank erfasst. Zu jedem Vorgang werden mindestens der Zeitpunkt, der Typ und die Kennung eines Objektes erfasst.
  \item Die Applikation muss Änderungen an Objekten erkennen und diese historisieren, also alle Änderungen in einen zeitlichen Zusammenhang bringen und ordnen können.
  \item Die Zieldatenbank soll für effiziente Analysen geeignet sein und Objekte effizient miteinander verknüpfen. Dadurch sollen möglichst schnell Verknüpfungen sowie ein Kontext hergestellt werden können.
  \item Die Zieldatenbank soll für sehr große Datenmengen konzipiert sein, da alle Objekte versioniert werden und somit mehrfach speichern werden.
\end{itemize}
\subsection{Nicht-funktionale Anforderungen}
Nicht-funktionale Anforderungen beschreiben den Aufbau des Systems. Auch sie werden in Form von Testfällen nach Beenden der Entwicklung geprüft und müssen für eine erfolgreiche Entwicklung erfüllt sein.\\
\begin{itemize}
\item Das System soll alle Extraktionsvorgänge in einer Datenbank dokumentieren und falls eine Ausführung nicht stattgefunden hat, dieses erkennen, diese bis zu einem gewissen Zeitpunkt nachholen. Eine Ausführung kann beispielsweise nicht stattfinden, wenn das System zum geplanten Zeitpunkt nicht lauffähig ist. Wird die Applikation später gestartet, soll diese mit der Ausführung beginnen.
\item Das Systen muss weitestgehend fehlerfrei arbeiten. Insbesondere müssen alle vorgesehenen Objekte extrahiert werden. Es darf kein Objekt, welches geändert oder neu erstellt wurde ohne Export im Quellsystem (durch z.B Fehler mit dem Zeitstempel durch Latenzen bei der Anfrage) verbleiben.
\item Das System muss eine hohe Fehlertoleranz aufweisen und verschiedene Mechanismen implementieren, welche eine hohe Resilienz des Systems gewährleisten. Falls in einem Extraktionsvorgang eine Ausnahme auftritt, so muss durch einen Logeintrag erkennbar sein, bei welchem Objekt und Objekttyp oder Jira-Schnittstelle diese verursacht wurde. Ein weiterer Versuch, das entsprechende Objekt zu extrahieren, soll beim nächsten regulären Extraktionsvorgang angestoßen werden.
\item Das System soll flexibel gestaltet sein und alle notwendigen Parameter zur Extraktion durch Umgebungsvariablen erhalten. Wichtige Parameter sind beispielsweise der Jira Benutzer, dessen Passwort und die URL der Jira Instanz. Ein Pfad zur Konfigurationsdatei aller relevanten Jira-Projekte sowie Vorgangstypen, welche zur Extraktion bestimmt sind, soll auch durch eine Umgebungsvariable gesetzt werden können.
\item Die Applikation soll speziell für containerisierte Umgebungen und Cloudumgebungen bestimmt sein und somit als Image für IaC-Anwendungen (z.B. Docker oder Kubernetes) ausführbar sein.
\item Die Applikation soll nicht skalierbar sein. Es ist nur eine Instanz für jeweils einen Extraktionsvorgang vorgesehen. Ein Entwurf, um das System als Cluster zu betreiben ist sinnvoll, wird aber aufgrund von übermäßigem Aufwand bei der Implementierung der Kommunikation zwischen Instanzen nicht umgesetzt.
\end{itemize}
Es werden keine Anforderungen an die Extraktionsdauer gestellt. Dies ergibt sich daraus, dass die Applikation als Hintergrundprozess ausgeführt wird und durch einen langen ETL-Prozess kein Anwender beeinträchtigt wird. Primäres Ziel der Applikation ist die Extraktion und Aufbereitung von fachlich korrekten und konsistenten Daten. Außerdem ist vorgesehen, die Anwendung gezielt auszuführen, wenn möglichst keine Anwender mit dem System interagieren.
